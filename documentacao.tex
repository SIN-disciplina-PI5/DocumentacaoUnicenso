\documentclass[a4paper,12pt]{article}
\usepackage{graphicx}
\usepackage{hyperref}
\usepackage{indentfirst}
\usepackage{float} % Adicione o pacote float

\begin{document}

\begin{titlepage}
  \centering
  \includegraphics[width=6cm]{logo.png} 
  
  \vspace*{2cm}

  \Huge
  \textbf{UNICENSO}

  \vspace{0.5cm}
  \LARGE
  A importância dos elementos visuais para a tomada de decisões

  \vspace{1.5cm}

  \small % Diminui o tamanho da fonte
  \textbf{ALMIR EVANGELISTA DE OLIVEIRA JUNIOR}\\
  \textbf{JULIO MATIAS DIAS DE SOUZA}\\
  \textbf{LETÍCIA VITÓRIA DOS SANTOS}\\
  \textbf{LUCAS BRAGA DE MEDEIROS}\\
  \textbf{WILLIAM VITOR DE MOURA ARAÚJO}\\
  \textbf{YASMIN PRISCILLA FERREIRA DO NASCIMENTO}

  \vfill

  \Large
  Trabalho realizado no curso de Sistemas para Internet pela Universidade Católica de Pernambuco, como requisito parcial para a obtenção de pontuação na disciplina de Projeto Integrador 5.

  \vspace{0.8cm}
  \Large
  Docente: Gabriel Fernandes Almeida

  \vspace{0.8cm}
  \Large
  RECIFE\\
  2024
\end{titlepage}



\newpage

\section*{AGRADECIMENTOS}

Gostaríamos de agradecer e ressaltar que todos os participantes da equipe do projeto UniCenso foram extremamente importantes para a realização e conclusão com êxito do trabalho. Cada participante teve seu papel na elaboração do projeto e conseguiu cumpri-lo. Além disso, gostaríamos de agradecer ao professor Gabriel, que trouxe muitos ensinamentos, ao mesmo tempo que corrigia e direcionava-nos ao que tinha que ser feito.

\vspace{2cm}

\begin{quote}
    \centering
    "Para realizar grandes conquistas, devemos não apenas agir, mas também sonhar; não apenas planejar, mas também acreditar." 

    (Anatole France)
\end{quote}

\section{RESUMO}
O seguinte ofício tem como objetivo apresentar um estudo sobre o impacto que as informações visuais podem causar na vida das pessoas que procuram e/ou fazem parte do Ensino Superior, ou seja, mostrar a relação que a inserção de gráficos para o meio educativo é de suma importância, bem como analisar as ligações entre as características pessoais e o ensino adquirido. O princípio deste levantamento se liga à descrição da problemática, e de como ela pode impulsionar modificações na concepção de mundo. A partir das informações descritas, foi realizada uma reflexão sobre o efeito dos dashboards - e o seu não uso -, mostrando o quanto o visual é indispensável na tomada de decisões tanto na vida acadêmica quanto no mercado de trabalho, como de modo geral. Além de que, se fez presente a exposição de dados acerca de como traços físicos, corroboram com resultados diferentes de participação no Ensino Superior ao longo dos anos.

\textbf{Palavras-chave:} dashboard, ensino superior, dados

\newpage

\tableofcontents

\newpage

\section{INTRODUÇÃO}
O presente trabalho visa apresentar pontuações sobre o desenvolvimento de dashboards referentes ao Ensino Superior, como também pretende explicitar informações visuais acerca de todos os cursos acadêmicos oferecidos. São objetivos deste ofício, resumir e explanar gráficos com os dados obtidos através do censo. A metodologia utilizada foi a pesquisa bibliográfica e a pesquisa quantitativa correlacional ligada ao sistema de consulta de base do censo dos cursos de graduação.






















