\documentclass[a4paper,12pt]{article}
\usepackage{graphicx}
\usepackage{hyperref}
\usepackage{indentfirst}
\usepackage{float} % Adicione o pacote float

\begin{document}

\begin{titlepage}
  \centering
  \includegraphics[width=6cm]{logo.png} 
  
  \vspace*{2cm}

  \Huge
  \textbf{UNICENSO}

  \vspace{0.5cm}
  \LARGE
  A importância dos elementos visuais para a tomada de decisões

  \vspace{1.5cm}

  \small % Diminui o tamanho da fonte
  \textbf{ALMIR EVANGELISTA DE OLIVEIRA JUNIOR}\\
  \textbf{JULIO MATIAS DIAS DE SOUZA}\\
  \textbf{LETÍCIA VITÓRIA DOS SANTOS}\\
  \textbf{LUCAS BRAGA DE MEDEIROS}\\
  \textbf{WILLIAM VITOR DE MOURA ARAÚJO}\\
  \textbf{YASMIN PRISCILLA FERREIRA DO NASCIMENTO}

  \vfill

  \Large
  Trabalho realizado no curso de Sistemas para Internet pela Universidade Católica de Pernambuco, como requisito parcial para a obtenção de pontuação na disciplina de Projeto Integrador 5.

  \vspace{0.8cm}
  \Large
  Docente: Gabriel Fernandes Almeida

  \vspace{0.8cm}
  \Large
  RECIFE\\
  2024
\end{titlepage}



\newpage

\section*{AGRADECIMENTOS}

Gostaríamos de agradecer e ressaltar que todos os participantes da equipe do projeto UniCenso foram extremamente importantes para a realização e conclusão com êxito do trabalho. Cada participante teve seu papel na elaboração do projeto e conseguiu cumpri-lo. Além disso, gostaríamos de agradecer ao professor Gabriel, que trouxe muitos ensinamentos, ao mesmo tempo que corrigia e direcionava-nos ao que tinha que ser feito.

\vspace{2cm}

\begin{quote}
    \centering
    "Para realizar grandes conquistas, devemos não apenas agir, mas também sonhar; não apenas planejar, mas também acreditar." 

    (Anatole France)
\end{quote}

\section{RESUMO}
O seguinte ofício tem como objetivo apresentar um estudo sobre o impacto que as informações visuais podem causar na vida das pessoas que procuram e/ou fazem parte do Ensino Superior, ou seja, mostrar a relação que a inserção de gráficos para o meio educativo é de suma importância, bem como analisar as ligações entre as características pessoais e o ensino adquirido. O princípio deste levantamento se liga à descrição da problemática, e de como ela pode impulsionar modificações na concepção de mundo. A partir das informações descritas, foi realizada uma reflexão sobre o efeito dos dashboards - e o seu não uso -, mostrando o quanto o visual é indispensável na tomada de decisões tanto na vida acadêmica quanto no mercado de trabalho, como de modo geral. Além de que, se fez presente a exposição de dados acerca de como traços físicos, corroboram com resultados diferentes de participação no Ensino Superior ao longo dos anos.

\textbf{Palavras-chave:} dashboard, ensino superior, dados

\newpage

\tableofcontents

\newpage

\section{INTRODUÇÃO}
O presente trabalho visa apresentar pontuações sobre o desenvolvimento de dashboards referentes ao Ensino Superior, como também pretende explicitar informações visuais acerca de todos os cursos acadêmicos oferecidos. São objetivos deste ofício, resumir e explanar gráficos com os dados obtidos através do censo. A metodologia utilizada foi a pesquisa bibliográfica e a pesquisa quantitativa correlacional ligada ao sistema de consulta de base do censo dos cursos de graduação.
\section{DESENVOLVIMENTO}
\subsection{Problema}
Ao se deparar com uma base de dados, é possível encontrar diversos erros, como valores vazios, duplicados, sem sentido, entre outros. Nesse sentido, faz-se necessário corrigir todos esses erros, para que posteriormente haja a possibilidade de criar os diversos dashboards referentes à base de dados utilizada no projeto. Com isso, percebe-se que a ausência de organização encontrada, e a falta de elementos visuais como gráficos e tabelas, podem causar várias dificuldades durante a interpretação e visualização dos dados.

A base de dados utilizada no projeto possui um grande volume com diversas linhas e várias colunas. Seguem alguns exemplos de colunas encontradas:
\begin{itemize}
    \item Nomes dos cursos
    \item Instituições
    \item Localização
    \item Raça
    \item Turno
    \item Modalidade de ensino
    \item Estudantes matriculados
    \item Ingressantes
    \item Estudantes concluintes
\end{itemize}

Embora esses dados sejam extremamente valiosos, a forma que eles se encontram originalmente pode ser difícil de compreender, quando se deseja realizar uma análise e interpretação.

\subsection{Solução}
Os dashboards criados e seus respectivos gráficos são uma forma eficaz de executar algumas ações importantes, tais como:
\begin{itemize}
    \item Visualização dos dados de forma clara e concisa
    \item Identificação de tendências e padrões
    \item Comparação de diferentes cursos e instituições
    \item Realização de perguntas e encontro de respostas
    \item Tomada de decisões mais informadas
\end{itemize}

Por exemplo, os dashboards que foram criados através do presente projeto, mostram diversas informações explícitas ou implícitas, como a distribuição dos cursos por área de estudo, a satisfação dos alunos com diferentes cursos, a demanda por profissionais em diferentes áreas, quantidade de ingressantes e concluintes por raça, quantidade de ingressantes por região, entre outros. Além disso, vale ressaltar que o trabalho desenvolvido conta com uma página de login, que ao realizá-lo o usuário é redirecionado para o dashboard, que além dos elementos visuais, também é possível encontrar filtros de ano, região, código da Instituição de Ensino Superior e código UF.

Esses recursos visuais gerados, podem ajudar em diferentes perspectivas como:
\begin{itemize}
    \item Auxiliar o estudante durante a escolha do curso superior
    \item Comparar diferentes opções de cursos e instituições
    \item Tornar as decisões tomadas mais precisas
    \item Ajudar durante a realização de pesquisas
\end{itemize}

Além disso, os dashboards podem ser usados por:
\begin{itemize}
    \item Instituições de ensino superior: para avaliar seus cursos e identificar áreas de melhoria.
    \item Governo: para formular políticas públicas de educação.
    \item Empresas: para identificar oportunidades de recrutamento.
\end{itemize}

Em resumo, percebe-se que os dashboards são de fato uma ferramenta poderosa que melhora de forma significativa a forma que os dados são utilizados, permitindo uma melhor compreensão e interpretação. Além disso, tornou-se perceptível a possibilidade de utilizá-los em diversos âmbitos.

\subsection{Onde a problemática se apresenta}
\subsubsection{Instituições de Ensino Superior:}
\begin{itemize}
    \item Dificuldade em acompanhar a demanda por informações por parte dos alunos.
    \item Falta de dados para embasar decisões estratégicas sobre os cursos.
    \item Dificuldade em avaliar a efetividade dos cursos.
\end{itemize}

\subsubsection{Estudantes:}
\begin{itemize}
    \item Dificuldade em encontrar informações confiáveis e comparáveis sobre cursos.
    \item Dificuldade em tomar decisões mais precisas sobre sua carreira.
    \item Possibilidade de escolher um curso que não seja adequado para seus objetivos.
\end{itemize}

\subsubsection{Governo:}
\begin{itemize}
    \item Dificuldade em formular políticas públicas de educação eficazes.
    \item Falta de dados para avaliar o impacto dos investimentos em educação superior.
\end{itemize}

\subsubsection{Empresa:}
\begin{itemize}
    \item Dificuldade em encontrar profissionais qualificados para as suas vagas.
    \item Falta de dados para identificar as habilidades necessárias para o futuro mercado de trabalho.
\end{itemize}






















